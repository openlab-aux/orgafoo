\documentclass{article}
\usepackage[german]{babel}

%%%%%%%%%% Start TeXmacs macros
\newcommand{\tmaffiliation}[1]{\\ #1}
\newcommand{\tmstrong}[1]{\textbf{#1}}
%%%%%%%%%% End TeXmacs macros

\begin{document}

\title{K{\"u}chennutzungsordnung}

\author{\tmaffiliation{Stand: 25.09.2015 (alpha1)}}

\maketitle


\begin{enumerate}
  \item Grunds{\"a}tzlich darf jedes Vereinsmitglied die K{\"u}che benutzen.
  
  Nicht-mitglieder d{\"u}rfen dies auch, nachdem sie Mitglied gefragt haben
  und die Antwort positiv war.
  
  \item Jeder Mensch der die K{\"u}che benutzt muss sie danach auch wieder
  aufr{\"a}umen.
  
  Also jeglichen von ihm/ihr verursachten Dreck, dreckiges Geschirr usw.
  aufr{\"a}umen, bzw. in die Sp{\"u}lmaschine r{\"a}umen.
  
  \item Weil das (also der Inhalt von Regel 2) \ in der Vergangenheit nur
  mittelm{\"a}{\ss}ig gut funktioniert hat, gibt es ab jetzt
  K{\"u}chennutzungstracking. Das bedeutet:
  \begin{enumerate}
    \item Jede Person $\mathcal{P}$ , die vorhat die K{\"u}che zu benutzen,
    hat sich {\tmstrong{vor}} der K{\"u}chennutzung \ in die Liste
    einzutragen. (Die Liste sollte neben diesem Dokument am K{\"u}hlschrank
    h{\"a}ngen, falls nicht an einer Schrankt{\"u}r oder so, auf jeden fall
    gut sichtbar.)
    
    \item Einzutragen sind die {\tmstrong{Namen}} (oder {\tmstrong{Nickname}},
    ids oder Pseudonyme sind auch okay) der Personen $\mathcal{P}$ (ich nehme
    mal an das manchmal Menschen zusammen kochen wollen),
    {\tmstrong{Zeitpunkt}} und das zubereitete {\tmstrong{Gericht}}.
    
    \item Was muss ich eintragen, was nicht?
    
    Grunds{\"a}tzlich gilt: Alles wobei Geschirr dreckig wird muss eingetragen
    werden.
    
    Beispiele:
    \begin{description}
      \begin{multicols}{2}
        \item[Gulasch kochen] Ja
        
        \item[Aufbackpizza] ja
        
        \item[Spezi holen] Nein
        
        \item[Kaffe machen] Grenzfall, hier definiert als: Nein
      \end{multicols}
    \end{description}
    \item Ich wei{\ss} diese Regeln sind sehr streng, auf Grund des Zustands
    der K{\"u}che in der Vergangenheit, m{\"o}chte ich das aber so streng.
    Wenn das Sauberhalten der K{\"u}che so gut l{\"a}uft , bin ich auch
    durchaus bereit diese Regel etwas lockerer zu machen.
  \end{enumerate}
  \item Einmal pro Woche (meistens am {\tmstrong{Wochenende}}, kann auch
  Montag werden) werde ich den Zustand der K{\"u}che kontrollieren und mit der
  Liste abgleichen. Ich habe dabei vor nicht allzu nazi m{\"a}{\ss}ig
  vorzugehen. Falls jedoch unaufger{\"a}umtes Zeug, Dreck, dreckiges Geschirr
  oder {\"A}hnliches vorhanden ist, werde ich verantwortliche Menschen dazu
  bringen ihren Kram aufzur{\"a}umen oder sie f{\"u}r zuk{\"u}nftige
  Putzaktionen als bevorzugte K{\"u}chenhelfer einplanen.
  
  \item Diese Nutzungsordnung (und auch der Vordruck, der tracking liste) sind
  gerade die Alpha, es kann durchaus sein, das ich in den n{\"a}chsten wochen
  noch dinge {\"a}ndere oder Verbesserungsvorschl{\"a}ge annehme, evtl. habe
  ich auch dinge vergessen. Verbesserungsvorschl{\"a}ge an mich (siehe 6.).
  
  \item F{\"u}r alle k{\"u}chenrelatierten Entscheidungen, Fragen und Foo bin
  vorerst ich verantwortlich, ich bin zu erreichen unter:
  
  \begin{multicols}{2}
    \begin{description}
      \item[irc] richi235 (freenode)
      
      \item[email] richard@weltraumpflege.org
    \end{description}
  \end{multicols}
\end{enumerate}

\end{document}
